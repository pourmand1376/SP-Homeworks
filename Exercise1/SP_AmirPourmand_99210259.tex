\documentclass{article}[12pt]
\usepackage{graphicx}
\usepackage{amsmath,amssymb}
\usepackage{tikz}
\usepackage{xepersian}
\settextfont[Scale=1]{IRXLotus}
\setlatintextfont[Scale=0.8]{}

\DeclareRobustCommand{\bbone}{\text{\usefont{U}{bbold}{m}{n}1}}

\DeclareMathOperator{\EX}{\mathbb{E}}% expected value
\let\P\relax
\DeclareMathOperator{\P}{\mathbb{P}}

\title{  \includegraphics[scale=0.35]{../logo.png} \\
    دانشکده مهندسی کامپیوتر
    \\
    دانشگاه صنعتی شریف
}
\author{استاد درس: دکتر حمیدرضا ربیعی}
\date{پاییز ۱۴۰۱}



\def \Subject {
تمرین اول
}
\def \Course {
درس یادگیری ماشین آماری
}
\def \Author {
نام و نام خانوادگی:
امیر پورمند}
\def \Email {\lr{pourmand1376@gmail.com}}
\def \StudentNumber {99210259}


\begin{document}

 \maketitle
 
\begin{center}
\vspace{.4cm}
{\bf {\huge \Subject}}\\
{\bf \Large \Course}
\vspace{.8cm}

{\bf \Author}

\vspace{0.3cm}

{\bf شماره دانشجویی: \StudentNumber}

\vspace{0.3cm}

آدرس ایمیل
:
{\bf \Email}
\end{center}


\clearpage
\section{سوال ۱}

زمان ارسال ایمیل برای علی تابع متغیر تصادفی 
$X$
خواهد بود که از توزیع پواسون با پارامتر لاندا پیروی میکند. 

\begin{equation}\begin{split}
	X \sim Pois(\lambda_A) \\
	P(X == 3 ) = \frac{\lambda^k e^{-\lambda}}{k!} = 
	\frac{\lambda_A^3 e^{-\lambda_A}}{3!}	 
	\end{split}
\end{equation}

	

 \end{document}