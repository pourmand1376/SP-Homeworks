\documentclass{article}[12pt]
\usepackage{graphicx}
\usepackage{amsmath,amssymb}
\usepackage{tikz}
\usepackage{xepersian}
\settextfont[Scale=1]{IRXLotus}
\setlatintextfont[Scale=0.8]{}

\DeclareRobustCommand{\bbone}{\text{\usefont{U}{bbold}{m}{n}1}}

\DeclareMathOperator{\EX}{\mathbb{E}}% expected value
\let\P\relax
\DeclareMathOperator{\P}{\mathbb{P}}

\title{  \includegraphics[scale=0.35]{../logo.png} \\
    دانشکده مهندسی کامپیوتر
    \\
    دانشگاه صنعتی شریف
}
\author{استاد درس: دکتر حمیدرضا ربیعی}
\date{پاییز ۱۴۰۱}



\def \Subject {
تمرین اول
}
\def \Course {
درس فرآیند تصادفی
}
\def \Author {
نام و نام خانوادگی:
امیر پورمند}
\def \Email {\lr{pourmand1376@gmail.com}}
\def \StudentNumber {99210259}


\begin{document}

 \maketitle
 
\begin{center}
\vspace{.4cm}
{\bf {\huge \Subject}}\\
{\bf \Large \Course}
\vspace{.8cm}

{\bf \Author}

\vspace{0.3cm}

{\bf شماره دانشجویی: \StudentNumber}

\vspace{0.3cm}

آدرس ایمیل
:
{\bf \Email}
\end{center}


\clearpage
\section{سوال ۱}
\subsection{الف}

زمان ارسال ایمیل برای علی تابع متغیر تصادفی 
$X$
خواهد بود که از توزیع پواسون با پارامتر لاندا پیروی میکند. 

\begin{equation}\begin{split}
	X \sim Pois(\lambda_A) \\
	P(X == 3) = \frac{\lambda^k e^{-\lambda}}{k!} = 
	\frac{\lambda_A^3 e^{-\lambda_A}}{3!}	 
	\end{split}
\end{equation}

\subsection{ب}
\subsubsection{الف}
میدانیم هر دو متغیر تصادفی دارای توزیع نمایی هستند که این توزیع عملا بدون حافظه است. یعنی رویداد دوم مستقل از زمان رویداد اول است. پس داریم:

\begin{equation}
	E \left[
	Y_2 | Y_1 
	 \right] = E [Y_2] = \frac{1}{\lambda_A}
\end{equation}
\subsubsection{ب}

میدانیم تابع چگانی 
$Y_1$
بصورت زیر است:
\begin{equation}
	Y_{1} =\begin{cases}
		\lambda e^{-\lambda } & Y_{1} \geqslant 0\\
		0 & Y_{1} < 0
	\end{cases}
\end{equation}
سپس داریم
\begin{equation}
	\begin{split}
		 \begin{array}{c}
			X\ =\ Y_{1}^{2} \ \Longrightarrow \ Y_{1} \ =\ \sqrt{x}\\
			f_{X}( x) \ =\ f_{Y_{1}}\left(\sqrt{x}\right)\frac{1}{2\sqrt{x}} \ =\lambda e^{-\lambda \sqrt{x}}\frac{1}{2\sqrt{x}} \ \ 
		\end{array}
	\end{split}
\end{equation}
\subsubsection{ج}	

\begin{equation}
	\begin{split}
		 \begin{array}{c}
			Y_{1} \ ||\ Y_{2} \ \\
			f_{Y_{1} ,Y_{2}}( y_{1} ,y_{2}) \ =\ f_{Y_{1}}( y_{1}) f_{Y_{2}}( y_{2}) \ =\ \lambda ^{2} e^{\lambda ( y_{1} +y_{2})}
		\end{array}
	\end{split}
\end{equation}


\subsection{ج}
\subsubsection{الف}
در اینجا امیدریاضی 
$\frac{1}{\lambda_A}$
است.
\subsubsection{ب}
در اینجا رویداد عملا اتفاق افتاده است پس امیدریاضی ۱ است!

\subsection{د}
\subsubsection{الف}
\subsubsection{ب}
\subsubsection{ج}

\subsection{ه}

\subsection{و}

\subsection{ز}

 \end{document}