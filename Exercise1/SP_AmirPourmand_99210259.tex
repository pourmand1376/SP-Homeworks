\documentclass{article}[12pt]
\usepackage{graphicx}
\usepackage{amsmath,amssymb}
\usepackage{tikz}
\usepackage{xepersian}
\settextfont[Scale=1]{IRXLotus}
\setlatintextfont[Scale=0.8]{}

\DeclareRobustCommand{\bbone}{\text{\usefont{U}{bbold}{m}{n}1}}

\DeclareMathOperator{\EX}{\mathbb{E}}% expected value
\let\P\relax
\DeclareMathOperator{\P}{\mathbb{P}}

\title{  \includegraphics[scale=0.35]{../logo.png} \\
    دانشکده مهندسی کامپیوتر
    \\
    دانشگاه صنعتی شریف
}
\author{استاد درس: دکتر حمیدرضا ربیعی}
\date{پاییز ۱۴۰۱}



\def \Subject {
تمرین اول
}
\def \Course {
درس فرآیند تصادفی
}
\def \Author {
نام و نام خانوادگی:
امیر پورمند}
\def \Email {\lr{pourmand1376@gmail.com}}
\def \StudentNumber {99210259}


\begin{document}

 \maketitle
 
\begin{center}
\vspace{.4cm}
{\bf {\huge \Subject}}\\
{\bf \Large \Course}
\vspace{.8cm}

{\bf \Author}

\vspace{0.3cm}

{\bf شماره دانشجویی: \StudentNumber}

\vspace{0.3cm}

آدرس ایمیل
:
{\bf \Email}
\end{center}


\clearpage
\section{سوال ۱}
\subsection{الف}

زمان ارسال ایمیل برای علی تابع متغیر تصادفی 
$X$
خواهد بود که از توزیع پواسون با پارامتر لاندا پیروی میکند. 

\begin{equation}\begin{split}
	X \sim Pois(\lambda_A) \\
	P(X == 3) = \frac{\lambda^k e^{-\lambda}}{k!} = 
	\frac{\lambda_A^3 e^{-\lambda_A}}{3!}	 
	\end{split}
\end{equation}

\subsection{ب}
\subsubsection{الف}

\begin{equation}
\begin{split}
		f_{Y_{2} |Y_{1}}( y_{2} |y_{1}) =f_{X_{2}}( y_{2} -y_{1}) =\lambda e^{-\lambda ( y_{2} -y_{1})}\\
		E[ Y_{2} |Y_{1}] =\int _{0}^{\infty } Y_{2} f_{Y_{2} |Y_{1}}( y_{2} |y_{1}) dy_{2} =\int _{0}^{\infty } Y_{2} \lambda e^{-\lambda ( y_{2} -y_{1})} dy_{2}\\
		=\frac{e^{\lambda y_{1}}}{\lambda }
\end{split}
\end{equation}


\subsubsection{ب}

میدانیم تابع چگانی 
$Y_1$
بصورت زیر است:
\begin{equation}
	Y_{1} =\begin{cases}
		\lambda e^{-\lambda } & Y_{1} \geqslant 0\\
		0 & Y_{1} < 0
	\end{cases}
\end{equation}
سپس داریم
\begin{equation}
	\begin{split}
		 \begin{array}{c}
			X\ =\ Y_{1}^{2} \ \Longrightarrow \ Y_{1} \ =\ \sqrt{x}\\
			f_{X}( x) \ =\ f_{Y_{1}}\left(\sqrt{x}\right)\frac{1}{2\sqrt{x}} \ =\lambda e^{-\lambda \sqrt{x}}\frac{1}{2\sqrt{x}} \ \ 
		\end{array}
	\end{split}
\end{equation}
\subsubsection{ج}	

\begin{equation}
	\begin{split}
		  \begin{array}{c}
		 	f_{Y_{1} ,Y_{2}}( y_{1} ,y_{2}) =f_{Y_{2} |Y_{1}}( y_{2} |y_{1}) f_{Y_{1}}( y_{1})\\
		 	=f_{X_{1}}( y_{2} -y_{1}) f_{Y_{1}}( y_{1}) =\lambda e^{-\lambda ( y_{2} -y_{1})} \lambda e^{-\lambda y_{1}} =\lambda ^{2} e^{-\lambda y_{2}}
		 \end{array}
	\end{split}
\end{equation}


\subsection{ج}
\subsubsection{الف}

\begin{equation}
	\begin{gathered}
				E[ Y_{1} |Y_{1} \leqslant 1] =?\\
				F_{Y_{1} |Y_{1} \leqslant 1}( y_{1}) =P( Y_{1} \leqslant y_{1} |Y_{1} \leqslant 1) =1-P( Y_{1} \geqslant y_{1} |Y_{1} \leqslant 1)\\
				=1-\frac{P( y_{1} \leqslant Y_{1} \leqslant 1)}{P( Y_{1} \leqslant 1)} =1-\frac{e^{-\lambda y_{1}} e^{-\lambda ( 1-y_{1})} \lambda ( 1-y_{1})}{e^{-\lambda } \lambda }\\
				=1-( 1-y_{1}) =y_{1}
	\end{gathered}
\end{equation}

\begin{equation}
	\begin{gathered}
		E[ Y_{1} |Y_{1} \leqslant 1] =\int _{0}^{1}( 1-y_{1}) dy_{1} =\left[ y_{1} -y_{1}^{2} /2\right]_{0}^{1} =\frac{1}{2}
	\end{gathered}
\end{equation}

\subsubsection{ب}
\begin{equation}
	E[ Y_{2} |Y_{1} \leqslant 1] =E[ 1+X_{2}] =1+\frac{1}{\lambda }
\end{equation}

\subsection{د}
\subsubsection{الف}
خب فرض کنیم علی 
$A$
نامه میفرستد در حالی که محمد 
$M$
نامه میفرستد. و نرخ ارسال ایمیل های علی را 
$\lambda_1=2\lambda_A$
و نرخ ارسال ایمیل های محمد را 
$\lambda_2=\lambda_B$
در نظر میگیریم و توجه داریم که با توجه به این که زمان علی و محمد متفاوت است در واقع 

 توزیع جمع نامه ها برابر خواهد بود با:


\begin{equation}
	\begin{aligned}
		p_{Z}(z) &=P(Z=z) \\
		&=\sum_{j=0}^{z} P(X=j \& Y=z-j) \quad \text { so } X+Y=z \\
		&=\sum_{j=0}^{z} P(X=j) P(Y=z-j)  \\
		&=\sum_{j=0}^{z} \frac{e^{-\lambda_{1}} \lambda_{1}^{j}}{j !} \frac{e^{-\lambda_{2}} \lambda_{2}^{z-j}}{(z-j) !} \\
		&=\sum_{j=0}^{z} \frac{1}{j !(z-j) !} e^{-\lambda_{1}} \lambda_{1}^{j} e^{-\lambda_{2}} \lambda_{2}^{z-j} \\
		&=\sum_{j=0}^{z} \frac{z !}{j !(z-j) !} \frac{e^{-\lambda_{1}} \lambda_{1}^{j} e^{-\lambda_{2}} \lambda_{2}^{z-j}}{z !} \\
		&=\sum_{j=0}^{z}\left(\begin{array}{l}
			z \\
			j
		\end{array}\right) \frac{e^{-\lambda_{1}} \lambda_{1}^{j} e^{-\lambda_{2}} \lambda_{2}^{z-j}}{z !} \\
		&=\frac{e^{-\lambda}}{z !} \sum_{j=0}^{z}\left(\begin{array}{l}
			z \\
			j
		\end{array}\right) \lambda_{1}^{j} \lambda_{2}^{z-j} \\
		&=\frac{e^{-\lambda}}{z !}\left(\lambda_{1}+\lambda_{2}\right)^{z}
	\end{aligned}
\end{equation}
حال توزیع نهایی برابر توزیع پواسون با پارامتر جمع این دو یعنی 
$\lambda_1+\lambda_2 = 2\lambda_A+\lambda_B$
خواهد بود. 
\subsubsection{ب}

\begin{equation}
	\begin{gathered}
			S_{0} =min( T_{0} ,1) \Longrightarrow \\
			F_{S_{0}}( s_{0}) =P( min( T_{0} ,1) < s_{0})\\
			=1-P( min( T_{0} ,1)  >s_{0}) =1-P( s_{0} < T_{0} ,\ s_{0} < 1)\\
			=1-P( s_{0} < T_{0}) P( s_{0} < 1) =1-II( s_{0} < 1)( 1-F_{T_{0}}( s_{0}))\\
			=1-\left( 1-\left( 1-e^{-\lambda s_{0}}\right)\right) =1-e^{-\lambda s_{0}}\\
			E[ S_{0}] =\int _{0}^{1} e^{-\lambda s_{0}} ds_{0} =\frac{-1}{\lambda }\left[ e^{-\lambda s_{0}}\right]_{0}^{1} =\frac{-1}{\lambda }\left( e^{-\lambda } -1\right) =\frac{1-e^{-\lambda }}{\lambda }\\
			E[ Q] =E[ S_{0}] +E[ T_{1}] =E[ min( T_{0} ,1)] +E[ exp( \lambda _{A})] =\frac{1}{\lambda_A } +\frac{1-e^{-\lambda_A }}{\lambda_A }
	\end{gathered}
\end{equation}

\subsubsection{ج}

\begin{equation}
	\begin{gathered}
			P( A=4|A+B=10) =P( A=4,B=6|A+B=10)\\
			=\frac{P( A=4) P( B=6)}{F_{A+B}( 10)} =\frac{e^{-2\lambda _{A}}( 2\lambda _{A})^{4}}{4!} \times \frac{e^{-\lambda _{B}} \lambda _{B}^{6}}{6!} \times \frac{10!}{e^{-2\lambda _{A} -\lambda _{B}}( 2\lambda _{A} +\lambda _{B})^{10}}\\
			=\frac{10*9*8*7}{4*3*2} \times \frac{2^{4} \lambda _{A}^{4} \lambda _{B}^{6}}{( 2\lambda _{A} +\lambda _{B})^{10}}\\
			=\frac{10*3*7*16*\lambda _{A}^{4} \lambda _{B}^{6}}{( 2\lambda _{A} +\lambda _{B})^{10}}
	\end{gathered}
\end{equation}

\subsection{ه}
\begin{equation}
	\begin{gathered}
				Chebyshev\ \\
				X\ \sim \ Pois( 4)\\
				E[ X] =4,\ Var( X) =4,\ \sigma ( X) =2\\
				P( |X-\mu |\geqslant k\sigma ) \leqslant \frac{1}{k^{2}}\\
				P( X\geqslant 5)\rightarrow P( X-4\geqslant 1)\\
				k\sigma =2k\leqslant 1\rightarrow k\leqslant \frac{1}{2}\rightarrow \frac{1}{k^{2}} \geqslant 4\ \rightarrow useless
	\end{gathered}
\end{equation}

\begin{equation}
	\begin{gathered}
			Markov\\
			X\sim Pois( 4)\\
			E[ X] =4\\
			P( X\geqslant a) \leqslant \frac{E[ X]}{a}\\
			P( X\geqslant 5) \leqslant \frac{4}{5} =0.8\ \rightarrow more\ useful
	\end{gathered}
\end{equation}
\subsection{و}
برای این مسئله باید ابتدا قضیه CLT را بیان کنم که میگوید اگر یک سری متغیر iid داشته باشیم توزیع میانگین این متغیرها همواره یک توزیع نرمال است. به طور دقیق تر برای این مثال وقتی 
$\lambda$
به اندازه کافی بزرگ باشد یعنی به اندازه 
$\lambda$
متغیرتصادفی داریم که هر کدام توزیع پواسون با پارامتر یک دارند و البته واریانس آنها نیز یک است. در اینجا جمع این متغیرهای پواسون همواره یک توزیع نرمال خواهد بود و برای تخمین میتوان از ان استفاده کرد. حال میتوان به جای حساب کردن توزیع پواسون در بازه صفر تا یک از توزیع نرمال با پارامتر لاندا به عنوان میانگین و البته واریانس لاندا برای این منظور استفاده کرد.
$N \sim N(\lambda,\lambda)$
\subsection{ز}

\begin{eqnarray*}
	\begin{gathered}
			\Lambda \sim Exp( 2)\\
			N|\Lambda \sim Pois( \lambda )\\
			P( N=n) =\int _{0}^{\infty } P( N=n|\Lambda =\lambda ) P( \Lambda =\lambda ) d\lambda \\
			=\int _{0}^{\infty }\frac{e^{-\lambda } \lambda ^{n}}{n!} \times 2e^{-2\lambda } d\lambda \\
			we\ know\ that\ \int _{0}^{\infty }\frac{e^{-x} x^{n}}{n!} dx=1\\
			=\frac{2}{3^{n}}\int _{0}^{\infty }\frac{e^{-3\lambda } 3^{n} \lambda ^{n}}{n!} d\lambda =\frac{2}{3^{n}}\\
			E\left[ N^{2}\right] =\int _{0}^{\infty } n^{2}\frac{2}{3^{n}} dn=\frac{4}{ln^{3}( 3)} 
	\end{gathered}
\end{eqnarray*}


\section{سوال ۲}
\subsection{الف}
\begin{equation}
	\begin{gathered}
			x( t) \ *\ h( t) =\int _{-\infty }^{+\infty } x( \tau ) h( t-\tau ) d\tau =\int _{-\infty }^{+\infty } e^{-a\tau } u( \tau ) e^{-b( t-\tau )} u( t-\tau ) d\tau \\
			=\int _{0}^{t} e^{-a\tau -bt+b\tau } d\tau =e^{-bt}\frac{1}{b-a}\left[ e^{( b-a) \tau }\right]_{0}^{t} =\frac{e^{-bt}}{b-a}\left[ e^{bt-at} -1\right] ,\ t >0
	\end{gathered}
\end{equation}
\subsection{ب}

\begin{equation}
	\begin{gathered}
			x( t) \ *\ h( t) =\int _{-\infty }^{+\infty } x( \tau ) h( t-\tau ) d\tau =\int _{-\infty }^{+\infty } e^{-a\tau } u( \tau ) u( t-\tau ) d\tau \\
			=\int _{0}^{t} e^{-a\tau } d\tau =\frac{-1}{a}\left[ e^{-a\tau }\right]_{0}^{t} =\frac{e^{-at} -1}{-a} ,\ t >0
	\end{gathered}
\end{equation}
\subsection{ج}
\begin{equation}
	\begin{gathered}
			x( t) \ *\ h( t) =\int _{-\infty }^{+\infty } x( \tau ) h( t-\tau ) d\tau =\int _{-\infty }^{+\infty } u( \tau ) u( t-\tau ) d\tau \\
			=\int _{0}^{t} d\tau =t,t >0
	\end{gathered}
\end{equation}


\section{سوال ۳}

\begin{equation}
	\begin{gathered}
		h( n) =\left(\frac{1}{5}\right)^{n} u[ n] ,\ \begin{cases}
			causal & n< 0\Longrightarrow h[ n] =0\\
			stable & \sum _{k=-\infty }^{+\infty } |h[ n] |=1+\frac{1}{5} +\frac{1}{25} +...\ =\ \frac{1}{1-\frac{1}{5}} =\frac{5}{4} < \infty 
		\end{cases}
	\end{gathered}
\end{equation}

\begin{equation}
	\begin{gathered}
		h( n) =0.8^{n} u( n+2) ,\begin{cases}
			noncausal & n=1,\ h[ n] \neq 0\\
			stable & 0.8^{-2} +0.8^{-1} +...=\frac{0.8^{-2}}{1-0.8} < \infty 
		\end{cases}
	\end{gathered}
\end{equation}

\begin{equation}
	\begin{gathered}
		h( n) =\left(\frac{1}{2}\right)^{n} u( -n) ,\begin{cases}
			noncausal & u< 0,h[ n]  >0\\
			not\ stable & 0.5^{0} +0.5^{-1} +...\ =\ \infty 
		\end{cases}
	\end{gathered}
\end{equation}

\begin{equation}
	\begin{gathered}
		h( n) =5^{n} u( 3-n) ,\begin{cases}
			noncausal & n=-1,h( n) \neq 0\\
			stable & 5^{3} +5^{2} +...\ =\frac{5^{3}}{1-\frac{1}{5}}
		\end{cases} < \infty 
	\end{gathered}
\end{equation}


\begin{equation}
	\begin{gathered}
		h( t) =e^{-4t} u( t-2) ,\begin{cases}
			casual & t< 0,h( t) =0\\
			stable & \int _{2}^{\infty } e^{-4t} dt=\frac{-1}{4}\left[ e^{-4t}\right]_{2}^{\infty } < \infty 
		\end{cases}
	\end{gathered}
\end{equation}


\begin{equation}
	\begin{gathered}
		h( t) =e^{-6t} u( 3-t) ,\begin{cases}
			noncausal & t=-1,h( t) \neq 0\\
			nonstable & \int _{-\infty }^{3} e^{-6t} dt=\infty 
		\end{cases}
	\end{gathered}
\end{equation}


\begin{equation}
	\begin{gathered}
		h( t) =e^{-2t} u( t+50) ,\begin{cases}
			noncausal & t=-1,h( t) \neq 0\\
			stable & \int _{-50}^{\infty } e^{-2t} dt=\frac{-1}{2}\left[ e^{-2t}\right]_{-50}^{\infty } < \infty 
		\end{cases}
	\end{gathered}
\end{equation}

 \end{document}